%\documentclass[handout,t]{beamer}
\documentclass{beamer}
\batchmode
%\usepackage[utf-8]{inputenc}
% \usepackage{pgfpages}
% \pgfpagesuselayout{4 on 1}[letterpaper,landscape,border shrink=5mm]


\usepackage{amsmath,amssymb,enumerate,epsfig,bbm,calc,color,ifthen,capt-of}

%\usepackage{beamerthemesplit}
\usetheme{Berlin}
\usecolortheme{bear}

\title{Methods for High-Order\\Particle-in-Cell Simulation}
\author[Jan S. Hesthaven and Andreas Kl\"ockner]{Jan S. Hesthaven\\Andreas Kl\"ockner}
\date[IAG Uni Stuttgart]{IAG Uni Stuttgart, Arbeitsgruppe Prof. Munz\\July 21, 2007}
\pgfdeclareimage[height=1cm]{brown-logo}{brown-logo.pdf}
\logo{\pgfuseimage{brown-logo}\hspace*{0.3cm}}

\def\bnabla{\mathbf \nabla}
\def\E{\mathbf E}
\def\H{\mathbf H}
\def\r{\mathbf r}
\def\k{\mathbf k}
\def\R{\mathbf R}
\def\K{\mathbf K}

\AtBeginSection[]
{
  \begin{frame}<beamer>
    \frametitle{Outline}
    \tableofcontents[currentsection]
  \end{frame}
}
\beamerdefaultoverlayspecification{<+->}

% -----------------------------------------------------------------------------
\newcommand{\assign}{:=}
\newcommand{\tmmathbf}[1]{\boldsymbol{#1}}
\newcommand{\tmop}[1]{\operatorname{#1}}
% -----------------------------------------------------------------------------
\newcommand{\discretized}[2]{{#1}^\triangle_{#2}}
\newcommand{\discretizedtwo}[2]{{#1}^{\triangle,2}_{#2}}
\newcommand{\anyip}[2]{\left\langle#1,#2\right\rangle}
\newcommand{\ip}[2]{\left\langle#1,#2\right\rangle_{\mathbb{R}^d}}
\newcommand{\ippc}[2]{\left\langle#1,#2\right\rangle_P}
\newcommand{\grad}{\tmmathbf{\nabla}}
\newcommand{\laplace}{\nabla^2}
\newcommand{\ipkgrad}[2]{\left\langle#1,#2\right\rangle_{\grad\Omega}}
\newcommand{\ipkgradmv}[2]{\left\langle#1,#2\right\rangle_{\grad\Omega}^{\text{MV}}}
\newcommand{\trp}[1]{{#1}^T}
\newcommand{\herm}[1]{{#1}^H}
\newcommand{\conj}[1]{{#1}^*}
\newcommand{\dimtimes}{\times}
\let\oldepsilon=\epsilon
\let\epsilon=\varepsilon

\newcommand{\identifier}[1]{{\tt#1}}
\newcommand{\inserttext}[1]{\verbatiminput{#1}}

\newcommand{\insertimage}[2]{\includegraphics[#1]{#2}}
% -----------------------------------------------------------------------------
\begin{document}
% -----------------------------------------------------------------------------

\frame{\titlepage}

\section[Outline]{}
\begin{frame}{Outline}
  \tableofcontents
\end{frame}

% -----------------------------------------------------------------------------
\section{Introduction}
\subsection{Why should and how can this be done?}
\begin{frame}{Why High-Order PIC?}
  \begin{itemize}
    \item Long Time Integration
    \item Improved Phase Error
    \item etc.
  \end{itemize}
  
\end{frame}
\begin{frame}{Why High-Order PIC?}
  \begin{itemize}
    \item Long Time Integration
    \item Improved Phase Error
    \item etc.
  \end{itemize}
  
\end{frame}
% -----------------------------------------------------------------------------
\section[Reconstruction]{Reconstructing the Current Density}
% -----------------------------------------------------------------------------
\section[Pushing]{Finding the per-Particle Forces}
% -----------------------------------------------------------------------------
\section[Resolution]{Resolution Management}
% -----------------------------------------------------------------------------
\section[Evaluating]{Evaluating Method Performance}
% -----------------------------------------------------------------------------
\section{Conclusions}
% -----------------------------------------------------------------------------
\end{document}
